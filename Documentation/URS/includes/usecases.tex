\section{Use cases}
\begin{usecase}
	
	\addtitle{Use Case 1}{Positioning a lane} 
	
	%Level: "user-goal" or "subfunction"
	\addfield{Level:}{User-goal}
	
	%Primary Actor: Calls on the system to deliver its services.
	\addfield{Primary Actor:}{End-User}
	
	%Preconditions: What must be true on start and worth telling the reader?
	\addfield{Preconditions:}{The application is open and no simulation is running.}
	
	%Main Success Scenario: A typical, unconditional happy path scenario of success.
	\addscenario{Main Success Scenario:}{
		\item User drags the lane from the sidebar.
		\item User places the lane on the grid.
		\item System sets the new placed lane and updates the grid.
	}
	
	%Extensions: Alternate scenarios of success or failure.
	\addscenario{Extensions:}{
		\item[2.a] If the space where the new lane is put is already engaged, then the system will ignore the new lane and goes back to step 1.
		\begin{enumerate}
			\item[1.] System informs end-user that the lane cannot be placed, because at that space already exists a lane.
			\item[2.] End of use case
		\end{enumerate}
		\item[2.b]If the lane is placed outside the grid, then the system will ignore the new lane and goes back to step 1.  
		\begin{enumerate}
			\item[1.] System informs end-user that the lane must be placed inside the grid.
				\item[2.] End of use case
		\end{enumerate}
	}
	
		\addfield{Post condition:}{The program stays in "Positioning a road" state. }
	
\end{usecase}

\begin{usecase}
	
	\addtitle{Use Case 2}{Rotating the lane} 
	
	%Level: "user-goal" or "subfunction"
	\addfield{Level:}{User-goal}
	
	%Primary Actor: Calls on the system to deliver its services.
	\addfield{Primary Actor:}{End-User}
	
	%Preconditions: What must be true on start and worth telling the reader?
	\addfield{Preconditions:}{There is at least one lane on the grid. No simulaion is running.}
	
	%Main Success Scenario: A typical, unconditional happy path scenario of success.
	\addscenario{Main Success Scenario:}{
		\item User right clicks on a lane.
		\item System shows a menu with several options
		\item User select "Rotate".
		\item System turns the lane 90 degrees.
		\item System updates the grid.
	}
	
	%Extensions: Alternate scenarios of success or failure.
	\addscenario{Extensions:}{
		\item[3] User does not want to execute any operation from the right click menu.
		\begin{enumerate}
			\item[1] User clicks on any space outside the right click menu area.
			\item[2] End of use case.
		\end{enumerate}
	}
	
		\addfield{Post condition:}{The program stays in "Rotating the lane" state. }
	
\end{usecase}

\begin{usecase}
	
	\addtitle{Use Case 3}{Positioning a crossing} 
	
	%Level: "user-goal" or "subfunction"
	\addfield{Level:}{User-goal}
	
	%Primary Actor: Calls on the system to deliver its services.
	\addfield{Primary Actor:}{End-User}
	
	%Preconditions: What must be true on start and worth telling the reader?
	\addfield{Preconditions:}{The application is open and no simulation is running.}
	
	%Main Success Scenario: A typical, unconditional happy path scenario of success.
	\addscenario{Main Success Scenario:}{
		\item User drags the crossing from the sidebar.
		\item User places the crossing on the grid.
	    \item System displays a pop-up window.
	    \item User sets the initial settings for the selected crossing.
		\item System sets the new placed lane and updates the grid.
	}
	
	%Extensions: Alternate scenarios of success or failure.
	\addscenario{Extensions:}{
		\item[2.a] If the space where the new crossing is put is already engaged, then the system will ignore the new crossing and goes back to step 1.
		\begin{enumerate}
			\item[1.] System informs end-user that the crossing cannot be placed, because at that space already exists a crossing.
			\item[2.] End of use case
		\end{enumerate}
		\item[2.b]If the crossing is placed outside the grid, then the system will ignore the new crossing and goes back to step 1.  
		\begin{enumerate}
			\item[1.] System informs end-user that the crossing must be placed inside the grid.
			\item[2.] End of use case
		\end{enumerate}
		\item[4.a] User does not set any of the required crossing attributes
		\begin{enumerate}
			\item A system default value will be applied to undefined attributes.
		\end{enumerate}
		\item [4.b] User does not want to set any settings.
		\begin{enumerate}
			\item [1.]User clicks on "Cancel" button.
			\item [2.]System closes the setting window.
		\end{enumerate}
	}
	
	\addfield{Post condition:}{The system displays the updated grid. }
	
	
\end{usecase}

\begin{usecase}
	
	\addtitle{Use Case 4}{Configurating traffic light timing} 
	
	%Level: "user-goal" or "subfunction"
	\addfield{Level:}{User-goal}
	
	%Primary Actor: Calls on the system to deliver its services.
	\addfield{Primary Actor:}{End-User}
	
	%Preconditions: What must be true on start and worth telling the reader?
	\addfield{Preconditions:}{There is at least one crossing on the grid and no simulation is running.}
	
	%Main Success Scenario: A typical, unconditional happy path scenario of success.
	\addscenario{Main Success Scenario:}{
		\item User right clicks on a crossing.
		\item System shows a menu with several options.
		\item User chooses "Traffic light configuration" from the menu.
		\item System pops up a new window with configuration options.
		\item User defines the amount of the time the traffic light is green.
		\item User clicks on the “OK” button.
		\item System closes the configuration window.
		\item System updates the crossing.
		
	}
	
	%Extensions: Alternate scenarios of success or failure.
	\addscenario{Extensions:}{
		\item[3.] User doesn not want to configurate anything. 
		\begin{enumerate}
			\item[1.] User clicks on "Cancel" button.
			\item[2.] System closes the configuration window.
			\item[3.] End of use case.
		\end{enumerate}
	}
	
	\addfield{Post condition:}{The system displays the updated grid. }
	
\end{usecase}

\begin{usecase}
	
	\addtitle{Use Case 5}{Deleting an crossing} 
	
	%Level: "user-goal" or "subfunction"
	\addfield{Level:}{User-goal}
	
	%Primary Actor: Calls on the system to deliver its services.
	\addfield{Primary Actor:}{End-User}
	
	%Preconditions: What must be true on start and worth telling the reader?
	\addfield{Preconditions:}{The application is open. There is at least one crossing on the grid. No simulation is running.}
	
	%Main Success Scenario: A typical, unconditional happy path scenario of success.
	\addscenario{Main Success Scenario:}{
		\item User right clicks on a crossing.
		\item System shows a menu of several options.
		\item User clicks on "Delete".
		\item System deletes the crossing and updates the grid.
	}
	
	%Extensions: Alternate scenarios of success or failure.
	\addscenario{Extensions:}{
		\item[3] User does not want to execute any operation from the right click menu.
		\begin{enumerate}
			\item[1] User clicks on any space outside the right click menu area.
			\item[2] End of use case.
		\end{enumerate}
	}
	
	
	
	\addfield{Post condition:}{The system displays the updated grid. }
	
\end{usecase}

\begin{usecase}
	
	\addtitle{Use Case 6}{Setting up a simulation} 
	
	%Level: "user-goal" or "subfunction"
	\addfield{Level:}{User-goal}
	
	%Primary Actor: Calls on the system to deliver its services.
	\addfield{Primary Actor:}{End-User}
	
	%Preconditions: What must be true on start and worth telling the reader?
	\addfield{Preconditions:}{There are components on the grid. No simulation is running.}
	
	%Main Success Scenario: A typical, unconditional happy path scenario of success.
	\addscenario{Main Success Scenario:}{
		\item User chooses setup to configure the simulation.
		\item System displays input boxes of for each lane.
		\item Users defines the amount of the cars coming through the lanes and fills in the input boxes.
		\item Users clicks on "Set up".
	}
	
	%Extensions: Alternate scenarios of success or failure.
	\addscenario{Extensions:}{
		\item[3] User does not want to do any settings.
		\begin{enumerate}
			\item[1] User clicks on "Cancel" button.
			\item[2] System hides the input boxes. 
			\item[3] End of use case.
	}
	
	\addfield{Post condition:}{The system is ready for running the simulation. }
\end{usecase}

\begin{usecase}
	
	\addtitle{Use Case 7}{Running a simulation} 
	
	%Level: "user-goal" or "subfunction"
	\addfield{Level:}{User-goal}
	
	%Primary Actor: Calls on the system to deliver its services.
	\addfield{Primary Actor:}{End-User}
	
	%Preconditions: What must be true on start and worth telling the reader?
	\addfield{Preconditions:}{The simulation is set up.}
	
	%Main Success Scenario: A typical, unconditional happy path scenario of success.
	\addscenario{Main Success Scenario:}{
		\item User clicks on "Play" button.
		\item System runs the simulation.
	}
	
	%Extensions: Alternate scenarios of success or failure.
	\addscenario{Extensions:}{
		\item[2] System detects errors.
		\begin{enumerate}
			\item[1.] System stops running the simulation and gives an error message.
		\end{enumerate}
	}
	
		\addfield{Post condition:}{The simulation is running. }
\end{usecase}

\begin{usecase}
	
	\addtitle{Use Case 8}{Stopping the simulation} 
	
	%Level: "user-goal" or "subfunction"
	\addfield{Level:}{User-goal}
	
	%Primary Actor: Calls on the system to deliver its services.
	\addfield{Primary Actor:}{End-User}
	
	%Preconditions: What must be true on start and worth telling the reader?
	\addfield{Preconditions:}{The system is running a simulation.}
	
	%Main Success Scenario: A typical, unconditional happy path scenario of success.
	\addscenario{Main Success Scenario:}{
		\item User clicks on "Stop" button.
		\item System stops the simulation.
	}
	
	\addfield{Post condition:}{The system is not running the simulation. }
	
\end{usecase}
\begin{usecase}
	
	\addtitle{Use Case 9}{Load file} 
	
	%Level: "user-goal" or "subfunction"
	\addfield{Level:}{User-goal}
	
	%Primary Actor: Calls on the system to deliver its services.
	\addfield{Primary Actor:}{End-User}
	
	%Preconditions: What must be true on start and worth telling the reader?
	\addfield{Preconditions:}{The application is open. No simulation is running. There is at least one saved file with traffic control system. }
	
	%Main Success Scenario: A typical, unconditional happy path scenario of success.
	\addscenario{Main Success Scenario:}{
		\item User selects file to open.
		\item System closes previous project.
		\item System loads the file.
	}
	
	%Extensions: Alternate scenarios of success or failure.
	\addscenario{Extensions:}{
		\item[2.a] File can't be loaded.
		\begin{enumerate}
			\item[1.] System informs user that file can't be loaded and given the choice to stop or choose another file.
			\item[2.] End of use case.
		\end{enumerate}
		\item[2.b] Another project is open
		\begin{enumerate}
			\item[1.] System asks if users wants, to save project, that is already open, before closing it and opening another project.
			\item[2.] End of use case.
		\end{enumerate}
	}
	
		\addfield{Post condition:}{The system has loaded an existing file. }
	
\end{usecase}
\begin{usecase}
	
	\addtitle{Use Case 10}{Save file} 
	
	%Level: "user-goal" or "subfunction"
	\addfield{Level:}{User-goal}
	
	%Primary Actor: Calls on the system to deliver its services.
	\addfield{Primary Actor:}{End-User}
	
	%Preconditions: What must be true on start and worth telling the reader?
	\addfield{Preconditions:}{The application is open. No simulation is running.}
	
	%Main Success Scenario: A typical, unconditional happy path scenario of success.
	\addscenario{Main Success Scenario:}{
		\item User click the "Save" button
		\item System saves the file.
	}
	%Extensions: Alternate scenarios of success or failure.
	\addscenario{Extensions:}{
		\item[2.a] Project file has not been saved on the device yet.
		\begin{enumerate}
			\item[1.] A saving dialogue window pops up.
			\item[2.] User chooses the directory where the file will be stored, and names the file.
			\item[3.] User clicks " OK ".
			\item[4.] System saves the file.
		\end{enumerate}
	}
	
		\addfield{Post condition:}{The system has saved a file. }
\end{usecase}

\begin{usecase}
	
	\addtitle{Use Case 11}{Save file as a new file} 
	
	%Level: "user-goal" or "subfunction"
	\addfield{Level:}{User-goal}
	
	%Primary Actor: Calls on the system to deliver its services.
	\addfield{Primary Actor:}{End-User}
	
	%Preconditions: What must be true on start and worth telling the reader?
	\addfield{Preconditions:}{ The application is open. Noo simulation is running.}
	
	%Main Success Scenario: A typical, unconditional happy path scenario of success.
	\addscenario{Main Success Scenario:}{
		\item User clicks " Save as " button.
		\item A saving dialogue window pops up.
		\item User chooses the directory where the file will be stored, and names the file.
		\item User clicks " OK ".
		\item System saves the file.
	}
	
	%Extensions: Alternate scenarios of success or failure.
	\addscenario{Extensions:}{
		\item[3.a] User mis-clicks the " Save as " button and does not want to save the file as a new file.
		\begin{enumerate}
			\item[1.] User clicks " Cancel ".
			\item[2.] Dialogue window closes.
		\end{enumerate}
	}
	
		\addfield{Post condition:}{The system has saved a file. }
\end{usecase}

\begin{usecase}
	
	\addtitle{Use Case 12}{Resizing the grid} 
	
	%Level: "user-goal" or "subfunction"
	\addfield{Level:}{User-goal}
	
	%Primary Actor: Calls on the system to deliver its services.
	\addfield{Primary Actor:}{End-User}
	
	%Preconditions: What must be true on start and worth telling the reader?
	\addfield{Preconditions:}{ The application is open. Noo simulation is running.}
	
	%Main Success Scenario: A typical, unconditional happy path scenario of success.
	\addscenario{Main Success Scenario:}{
		\item User clicks on “Edit” on the top bar.
		\item System displays a menu with several options.
		\item User clicks on "Document settings".
		\item System shows document settings panel.
		\item User defines the size of the grid by tying numbers in the width and height input boxes.
		\item User clicks on “OK”.
		\item System updates and display the new grid.
	}
	
	%Extensions: Alternate scenarios of success or failure.
	\addscenario{Extensions:}{
		\item[5] Use wants to make the grid size smaller. 
		\begin{enumerate}
			\item  System discards the objects placed outside of the new grid size. 
		\end{enumerate}
	}
	
	\addfield{Post condition:}{System displays the resized grid.}
\end{usecase}

